\documentclass[a4paper, 12pt, fleqn]{article}
\usepackage[T2A, T1]{fontenc}
\usepackage[utf8]{inputenc}
\usepackage[main=russian, english]{babel}
\usepackage{indentfirst}
\usepackage{amsmath}
\usepackage{amssymb}
\usepackage{geometry}
\usepackage{graphicx}
\usepackage{listings}
\usepackage{xcolor}
\definecolor{codegreen}{rgb}{0,0.6,0}
\definecolor{codegray}{rgb}{0.5,0.5,0.5}
\definecolor{codepurple}{rgb}{0.58,0,0.82}
\definecolor{backcolour}{rgb}{0.95,0.95,0.92}
\lstdefinestyle{mystyle}{
backgroundcolor=\color{backcolour},  
   commentstyle=\color{codegreen},
   keywordstyle=\color{magenta},
   numberstyle=\tiny\color{codegray},
   stringstyle=\color{codepurple},
   basicstyle=\ttfamily\footnotesize,
   breakatwhitespace=false,        
   breaklines=true,                
   captionpos=b,                  
   keepspaces=true,                
   numbers=left,                  
   numbersep=5pt,                
   showspaces=false,               
   showstringspaces=false,
   showtabs=false,                
   tabsize=2
}
\lstset{style=mystyle}
\geometry{margin=1in}
\begin{document}
\begin{titlepage}
\begin{figure}[!htb]
\centering
\includegraphics[width=0.04\textwidth]{fefu_logo.jpg}
\end{figure}
\begin{center}
\fontsize{10pt}{18pt}\selectfont
МИНИСТЕРСТВО НАУКИ И ВЫСШЕГО ОБРАЗОВАНИЯ И НАУКИ РОССИЙСКОЙ ФЕДЕРАЦИИ

Федеральное государственное автономное образовательное учреждение высшего образования

\textbf{«Дальневосточный федеральный университет»}

(ДВФУ)

\end{center}
\vspace{2mm}
\noindent\rule{\textwidth}{2pt}
\vspace{2mm}
\begin{center}
\textbf{ИНСТИТУТ МАТЕМАТИКИ И КОМПЬЮТЕРНЫХ ТЕХНОЛОГИЙ}

\vspace{12mm}
\textbf{Департамент математического и компьютерного моделирования}

\vspace{12mm}
\textbf{ЛАБОРАТОРНАЯ РАБОТА №3}

\vspace{3mm}
По основной образовательной программе подготовки бакалавров направлению 02.03.01 Математика и компьютерне науки профиль «Сквозные цифровые технологии»
\vspace{9mm}
\end{center}
\hfill
\begin{minipage}[t]{0.5\textwidth}
\raggedright
Студент группы Б9121-02.03.01сцт

\vspace{2mm}
\underline{Канцуров Александр Вадимович}

\vspace{2mm}
«19» январь 2024 г.

\vspace{24pt}
Преподаватель, \underline{кандидат физико-} \underline{математических наук\phantom{qqqqqqqqqqq}}

\vspace{2mm}
\underline{Яковлев Анатолий Александрович}

\vspace{2mm}
\underline{\phantom{s}\hspace{5cm}} (подпись)

\vspace{2mm}
\underline{Яковлев Анатолий Александрович}

\vspace{2mm}
«\underline{\phantom{dddd}\hspace{0.5mm}}» \underline{\phantom{}\hspace{3cm}}2024 г.

\end{minipage}

\vspace{24mm}
\begin{center}
г. Владивосток

2024
\end{center}
\end{titlepage}
\section*{Постановка задачи}
Дана задача:
\setcounter{MaxMatrixCols}{22}
\[\left\{\begin{aligned}
&c \cdot x \rightarrow \max\\
&Ax \le b\\
&x \ge 0
\end{aligned}
\right.
\]

Где \(c\) - неотрицательный 6-мерный вектор, \(x\) - неотрицательный6-мерный вектор неизвестных, который необходимо найти, \(A\) - матрица 8x6, \(b\) - неотрицательный 8-мерный вектор
\[
A =\scalebox{1}{
\setlength{\arraycolsep}{2pt}
\renewcommand{\arraystretch}{1}
\begin{pmatrix}
180  & 192  & 226  & 253  & 181  & 145  \\
75  & 226  & 122  & 195  & 115  & 142  \\
241  & 146  & 175  & 67  & 79  & 128  \\
266  & 162  & 252  & 42  & 278  & 67  \\
297  & 198  & 209  & 98  & 280  & 57  \\
169  & 204  & 132  & 160  & 164  & 232  \\
12  & 267  & 236  & 202  & 158  & 261  \\
296  & 173  & 55  & 85  & 285  & 279 
\end{pmatrix}
}
\]

\[
b =\scalebox{1}{
\setlength{\arraycolsep}{2pt}
\renewcommand{\arraystretch}{1}
\begin{pmatrix}
158  \\
170  \\
215  \\
256  \\
143  \\
180  \\
295  \\
87 
\end{pmatrix}
}
\]

\[
c =\scalebox{1}{
\setlength{\arraycolsep}{2pt}
\renewcommand{\arraystretch}{1}
\begin{pmatrix}
21  & 36  & 50  & 243  & 156  & 206 
\end{pmatrix}
}
\]


Решать будем симплекс-методом. Для начала приведем задачу к каноническому виду. Введем дополнительный 8-мерный вектор переменных \(z = Ax - b\)

Тогда к вектору c дописываем 8 нулей и рассматриваем вектор \(\begin{pmatrix}x\\z\end{pmatrix}\). К матрице \(A\) справа дописываем единичную матрицу получаем:

\[\left\{\begin{aligned}
&(c,0) \cdot \begin{pmatrix}x\\ z \end{pmatrix} \rightarrow \max\\
&(AI)\cdot\begin{pmatrix}x\\ z \end{pmatrix} = b\\
&x,z \ge 0
\end{aligned}
\right.
\]

\section*{Прямая задача}
Составим симплекс-таблицу. Первая строка – расширенный вектор \(c\), где элементы мы запишем со знаком минус, чтобы решать задачу на минимум. Остальные строки – расширенная матрица \(A\), последний столбик – вектор \(b\), а первый элемент последнего столбца - значение целевой функции, равное 0.

Видим, что в первой строке (не включая значение целевой функции) есть отрицательные элементы, а значит оптимальное решение еще не найдено.

\underline{Разрешающая колонка} находится путем выборки такого столбца, у которого элемент строки целевой функции отрицательный. Мы будем брать отрицательный элемент, максимальный по модулю.

\underline{Разрешающей строкой} будет строка, содержащая наименьшее положительное отношение свободного числа к элементу разрешающего столбца.

Элемент, расположенный на пересечении разрешающих столбцаи строки, называется разрешающим элементом

\[
\scalebox{1}{
\setlength{\arraycolsep}{2pt}
\renewcommand{\arraystretch}{1}
\begin{pmatrix}
-21.0  & -36.0  & -50.0  & -243.0  & -156.0  & -206.0  & 0.0  & 0.0  & 0.0  & 0.0  & 0.0  & 0.0  & 0.0  & 0.0  & 0.0  \\
180.0  & 192.0  & 226.0  & \colorbox{yellow}{253.0 }  & 181.0  & 145.0  & 1.0  & 0.0  & 0.0  & 0.0  & 0.0  & 0.0  & 0.0  & 0.0  & 158.0  \\
75.0  & 226.0  & 122.0  & 195.0  & 115.0  & 142.0  & 0.0  & 1.0  & 0.0  & 0.0  & 0.0  & 0.0  & 0.0  & 0.0  & 170.0  \\
241.0  & 146.0  & 175.0  & 67.0  & 79.0  & 128.0  & 0.0  & 0.0  & 1.0  & 0.0  & 0.0  & 0.0  & 0.0  & 0.0  & 215.0  \\
266.0  & 162.0  & 252.0  & 42.0  & 278.0  & 67.0  & 0.0  & 0.0  & 0.0  & 1.0  & 0.0  & 0.0  & 0.0  & 0.0  & 256.0  \\
297.0  & 198.0  & 209.0  & 98.0  & 280.0  & 57.0  & 0.0  & 0.0  & 0.0  & 0.0  & 1.0  & 0.0  & 0.0  & 0.0  & 143.0  \\
169.0  & 204.0  & 132.0  & 160.0  & 164.0  & 232.0  & 0.0  & 0.0  & 0.0  & 0.0  & 0.0  & 1.0  & 0.0  & 0.0  & 180.0  \\
12.0  & 267.0  & 236.0  & 202.0  & 158.0  & 261.0  & 0.0  & 0.0  & 0.0  & 0.0  & 0.0  & 0.0  & 1.0  & 0.0  & 295.0  \\
296.0  & 173.0  & 55.0  & 85.0  & 285.0  & 279.0  & 0.0  & 0.0  & 0.0  & 0.0  & 0.0  & 0.0  & 0.0  & 1.0  & 87.0 
\end{pmatrix}
}
\]

Начальное угловое решение:

\[
\scalebox{1}{
\setlength{\arraycolsep}{2pt}
\renewcommand{\arraystretch}{1}
\begin{pmatrix}
0.0  & 0.0  & 0.0  & 0.0  & 0.0  & 0.0  & 158.0  & 170.0  & 215.0  & 256.0  & 143.0  & 180.0  & 295.0  & 87.0 
\end{pmatrix}
}
\]

разрешающий столбец = 4

разрешающая строка = 2

разрешающий элемент = 253.0

Преобразовываем строки матрицы, то есть один из базисных столбцов станет \textbf{не} базисным, а разрешающий столбец – базисным:

\begin{enumerate}
\item Элементы разрешающей строки делим на разрешающий элемент, так как разрешающий элемент = 1, то строка останется прежней.
\item Преобразования остальных строк: Новая строка = Строка – элемент строки в разрешающем столбце * элемент разрешающей строки
\end{enumerate}

\[
\scalebox{1}{
\setlength{\arraycolsep}{2pt}
\renewcommand{\arraystretch}{1}
\begin{pmatrix}
151.89  & 148.41  & 167.07  & 0.0  & 17.85  & -66.73  & 0.96  & 0.0  & 0.0  & 0.0  & 0.0  & 0.0  & 0.0  & 0.0  & 151.75  \\
0.71  & 0.76  & 0.89  & 1.0  & 0.72  & 0.57  & 0.0  & 0.0  & 0.0  & 0.0  & 0.0  & 0.0  & 0.0  & 0.0  & 0.62  \\
-63.74  & 78.02  & -52.19  & 0.0  & -24.51  & 30.24  & -0.77  & 1.0  & 0.0  & 0.0  & 0.0  & 0.0  & 0.0  & 0.0  & 48.22  \\
193.33  & 95.15  & 115.15  & 0.0  & 31.07  & 89.6  & -0.26  & 0.0  & 1.0  & 0.0  & 0.0  & 0.0  & 0.0  & 0.0  & 173.16  \\
236.12  & 130.13  & 214.48  & 0.0  & 247.95  & 42.93  & -0.17  & 0.0  & 0.0  & 1.0  & 0.0  & 0.0  & 0.0  & 0.0  & 229.77  \\
227.28  & 123.63  & 121.46  & 0.0  & 209.89  & 0.83  & -0.39  & 0.0  & 0.0  & 0.0  & 1.0  & 0.0  & 0.0  & 0.0  & 81.8  \\
55.17  & 82.58  & -10.92  & 0.0  & 49.53  & 140.3  & -0.63  & 0.0  & 0.0  & 0.0  & 0.0  & 1.0  & 0.0  & 0.0  & 80.08  \\
-131.72  & 113.7  & 55.56  & 0.0  & 13.49  & 145.23  & -0.8  & 0.0  & 0.0  & 0.0  & 0.0  & 0.0  & 1.0  & 0.0  & 168.85  \\
235.53  & 108.49  & -20.93  & 0.0  & 224.19  & \colorbox{yellow}{230.28 }  & -0.34  & 0.0  & 0.0  & 0.0  & 0.0  & 0.0  & 0.0  & 1.0  & 33.92 
\end{pmatrix}
}
\]

В первой строке (не включая значение целевой функции) есть отрицательные элементы, а значит оптимальное решение еще не найдено

разрешающий столбец = 6

разрешающая строка = 9

разрешающий элемент = 230.28458498023716

Преобразовываем строки матрицы
\[
\scalebox{1}{
\setlength{\arraycolsep}{2pt}
\renewcommand{\arraystretch}{1}
\begin{pmatrix}
220.14  & 179.85  & 161.0  & 0.0  & 82.81  & 0.0  & 0.86  & 0.0  & 0.0  & 0.0  & 0.0  & 0.0  & 0.0  & 0.29  & 161.58  \\
0.13  & 0.49  & 0.95  & 1.0  & 0.16  & 0.0  & 0.0  & 0.0  & 0.0  & 0.0  & 0.0  & 0.0  & 0.0  & -0.0  & 0.54  \\
-94.66  & 63.77  & -49.44  & 0.0  & -53.95  & 0.0  & -0.73  & 1.0  & 0.0  & 0.0  & 0.0  & 0.0  & 0.0  & -0.13  & 43.77  \\
101.69  & 52.94  & 123.29  & 0.0  & -56.16  & 0.0  & -0.13  & 0.0  & 1.0  & 0.0  & 0.0  & 0.0  & 0.0  & -0.39  & 159.96  \\
192.21  & 109.9  & 218.38  & 0.0  & 206.16  & 0.0  & -0.1  & 0.0  & 0.0  & 1.0  & 0.0  & 0.0  & 0.0  & -0.19  & 223.45  \\
226.42  & 123.24  & 121.53  & 0.0  & 209.08  & 0.0  & -0.39  & 0.0  & 0.0  & 0.0  & 1.0  & 0.0  & 0.0  & -0.0  & 81.68  \\
-88.33  & 16.48  & 1.83  & 0.0  & -87.05  & 0.0  & -0.43  & 0.0  & 0.0  & 0.0  & 0.0  & 1.0  & 0.0  & -0.61  & 59.42  \\
-280.25  & 45.28  & 68.76  & 0.0  & -127.9  & 0.0  & -0.59  & 0.0  & 0.0  & 0.0  & 0.0  & 0.0  & 1.0  & -0.63  & 147.46  \\
1.02  & 0.47  & -0.09  & 0.0  & 0.97  & 1.0  & -0.0  & 0.0  & 0.0  & 0.0  & 0.0  & 0.0  & 0.0  & 0.0  & 0.15 
\end{pmatrix}
}
\]

В первой строке (не включая значение целевой функции) НЕТ отрицательных элементов, а значит оптимальное решение найдено


\underline{Оптимальное решение:}

\[
\scalebox{1}{
\setlength{\arraycolsep}{2pt}
\renewcommand{\arraystretch}{1}
\begin{pmatrix}
0.0  & 0.0  & 0.0  & 0.54  & 0.0  & 0.15 
\end{pmatrix}
}
\]

\underline{Целевая функция} = 161.58331330884624

\section*{Двойственная задача}
Двойственная задача будет выглядеть так:
\[\left\{\begin{aligned}
&b \cdot x \rightarrow \min\\
&A^{T}y \ge c\\
&y \ge 0
\end{aligned}
\right.
\]

Где \(c\) - неотрицательный 6-мерный вектор, \(y\) - неотрицательный8-мерный вектор неизвестных, который необходимо найти, \(A^{T}\) - матрица 6x8, \(b\) - неотрицательный 8-мерный вектор

\[
A^{T} =\scalebox{1}{
\setlength{\arraycolsep}{2pt}
\renewcommand{\arraystretch}{1}
\begin{pmatrix}
180  & 75  & 241  & 266  & 297  & 169  & 12  & 296  \\
192  & 226  & 146  & 162  & 198  & 204  & 267  & 173  \\
226  & 122  & 175  & 252  & 209  & 132  & 236  & 55  \\
253  & 195  & 67  & 42  & 98  & 160  & 202  & 85  \\
181  & 115  & 79  & 278  & 280  & 164  & 158  & 285  \\
145  & 142  & 128  & 67  & 57  & 232  & 261  & 279 
\end{pmatrix}
}
\]

\[
b =\scalebox{1}{
\setlength{\arraycolsep}{2pt}
\renewcommand{\arraystretch}{1}
\begin{pmatrix}
158  & 170  & 215  & 256  & 143  & 180  & 295  & 87 
\end{pmatrix}
}
\]

\[
c =\scalebox{1}{
\setlength{\arraycolsep}{2pt}
\renewcommand{\arraystretch}{1}
\begin{pmatrix}
21  \\
36  \\
50  \\
243  \\
156  \\
206 
\end{pmatrix}
}
\]


Для начала приведем задачу к каноническому виду. Введем дополнительный 6-мерный вектор переменных \(z = Ax - b\)

Тогда к вектору c дописываем 6 нулей и рассматриваем вектор \(\begin{pmatrix}y\\z\end{pmatrix}\). К матрице \(A\) справа дописываем единичную матрицу со знаком минус, получаем:

\[\left\{\begin{aligned}
&(c,0) \cdot \begin{pmatrix}y\\ z \end{pmatrix} \rightarrow \min\\
&(A^{T}(-I))\cdot\begin{pmatrix}y\\ z \end{pmatrix} = c\\
&y,z \ge 0
\end{aligned}
\right.
\]

Двойственная задача не имеет начального углового решения, что бы его найти необходимо решить вспомогательную задачу. Введем неотрицательный 8-мерный вектор \(u\), тогда получим равенство \(Ax+u = b\) и будем решать задачу не на наш минимум (начальный), а на сумму компонент вектора \(u\), получим
\[\left\{\begin{aligned}
&\sum_{i=1}^{m} u_i \rightarrow \min\\
&(A^{T}(-I)I)\cdot\begin{pmatrix}y\\ z \\u \end{pmatrix} = c\\
&y,z \ge 0,~u\ge0
\end{aligned}
\right.
\]

И в качестве начальной точки для этой задачи рассмотрим \(x = 0\), а \(u = b\). Решаем симплекс-методом и если решение \(u = 0\), то тогда мы получим точку \(x\), для которой \(x = b, ~x \ge 0\) и оно допустимое.
\section*{Вспомогательная задача}
Составим симплекс-таблицу
\[
\scalebox{0.7}{
\setlength{\arraycolsep}{2pt}
\renewcommand{\arraystretch}{1}
\begin{pmatrix}
0.0  & 0.0  & 0.0  & 0.0  & 0.0  & 0.0  & 0.0  & 0.0  & 0.0  & 0.0  & 0.0  & 0.0  & 0.0  & 0.0  & 1.0  & 1.0  & 1.0  & 1.0  & 1.0  & 1.0  & 0.0  \\
180.0  & 75.0  & 241.0  & 266.0  & 297.0  & 169.0  & 12.0  & 296.0  & -1.0  & -0.0  & -0.0  & -0.0  & -0.0  & -0.0  & 1.0  & 0.0  & 0.0  & 0.0  & 0.0  & 0.0  & 21.0  \\
192.0  & 226.0  & 146.0  & 162.0  & 198.0  & 204.0  & 267.0  & 173.0  & -0.0  & -1.0  & -0.0  & -0.0  & -0.0  & -0.0  & 0.0  & 1.0  & 0.0  & 0.0  & 0.0  & 0.0  & 36.0  \\
226.0  & 122.0  & 175.0  & 252.0  & 209.0  & 132.0  & 236.0  & 55.0  & -0.0  & -0.0  & -1.0  & -0.0  & -0.0  & -0.0  & 0.0  & 0.0  & 1.0  & 0.0  & 0.0  & 0.0  & 50.0  \\
253.0  & 195.0  & 67.0  & 42.0  & 98.0  & 160.0  & 202.0  & 85.0  & -0.0  & -0.0  & -0.0  & -1.0  & -0.0  & -0.0  & 0.0  & 0.0  & 0.0  & 1.0  & 0.0  & 0.0  & 243.0  \\
181.0  & 115.0  & 79.0  & 278.0  & 280.0  & 164.0  & 158.0  & 285.0  & -0.0  & -0.0  & -0.0  & -0.0  & -1.0  & -0.0  & 0.0  & 0.0  & 0.0  & 0.0  & 1.0  & 0.0  & 156.0  \\
145.0  & 142.0  & 128.0  & 67.0  & 57.0  & 232.0  & 261.0  & 279.0  & -0.0  & -0.0  & -0.0  & -0.0  & -0.0  & -1.0  & 0.0  & 0.0  & 0.0  & 0.0  & 0.0  & 1.0  & 206.0 
\end{pmatrix}
}
\]

Выделим базисные столбцы с помощью элементарных преобразований строк. К первой строке добавим все остальные строки, умноженные на -1. Получаем:

\[
\scalebox{0.7}{
\setlength{\arraycolsep}{2pt}
\renewcommand{\arraystretch}{1}
\begin{pmatrix}
-1177.0  & -875.0  & -836.0  & -1067.0  & -1139.0  & -1061.0  & -1136.0  & -1173.0  & 1.0  & 1.0  & 1.0  & 1.0  & 1.0  & 1.0  & 0.0  & 0.0  & 0.0  & 0.0  & 0.0  & 0.0  & -712.0  \\
\colorbox{yellow}{180.0 }  & 75.0  & 241.0  & 266.0  & 297.0  & 169.0  & 12.0  & 296.0  & -1.0  & -0.0  & -0.0  & -0.0  & -0.0  & -0.0  & 1.0  & 0.0  & 0.0  & 0.0  & 0.0  & 0.0  & 21.0  \\
192.0  & 226.0  & 146.0  & 162.0  & 198.0  & 204.0  & 267.0  & 173.0  & -0.0  & -1.0  & -0.0  & -0.0  & -0.0  & -0.0  & 0.0  & 1.0  & 0.0  & 0.0  & 0.0  & 0.0  & 36.0  \\
226.0  & 122.0  & 175.0  & 252.0  & 209.0  & 132.0  & 236.0  & 55.0  & -0.0  & -0.0  & -1.0  & -0.0  & -0.0  & -0.0  & 0.0  & 0.0  & 1.0  & 0.0  & 0.0  & 0.0  & 50.0  \\
253.0  & 195.0  & 67.0  & 42.0  & 98.0  & 160.0  & 202.0  & 85.0  & -0.0  & -0.0  & -0.0  & -1.0  & -0.0  & -0.0  & 0.0  & 0.0  & 0.0  & 1.0  & 0.0  & 0.0  & 243.0  \\
181.0  & 115.0  & 79.0  & 278.0  & 280.0  & 164.0  & 158.0  & 285.0  & -0.0  & -0.0  & -0.0  & -0.0  & -1.0  & -0.0  & 0.0  & 0.0  & 0.0  & 0.0  & 1.0  & 0.0  & 156.0  \\
145.0  & 142.0  & 128.0  & 67.0  & 57.0  & 232.0  & 261.0  & 279.0  & -0.0  & -0.0  & -0.0  & -0.0  & -0.0  & -1.0  & 0.0  & 0.0  & 0.0  & 0.0  & 0.0  & 1.0  & 206.0 
\end{pmatrix}
}
\]

Начальное угловое решение

\[
\scalebox{1}{
\setlength{\arraycolsep}{2pt}
\renewcommand{\arraystretch}{1}
\begin{pmatrix}
0.0  & 0.0  & 0.0  & 0.0  & 0.0  & 0.0  & 0.0  & 0.0  & 0.0  & 0.0  & 0.0  & 0.0  & 0.0  & 0.0  & 21.0  & 36.0  & 50.0  & 243.0  & 156.0  & 206.0 
\end{pmatrix}
}
\]

разрешающий столбец = 1

разрешающая строка = 2

разрешающий элемент = 180.0

\[
\scalebox{0.7}{
\setlength{\arraycolsep}{2pt}
\renewcommand{\arraystretch}{1}
\begin{pmatrix}
0.0  & -384.58  & 739.87  & 672.34  & 803.05  & 44.07  & -1057.53  & 762.51  & -5.54  & 1.0  & 1.0  & 1.0  & 1.0  & 1.0  & 6.54  & 0.0  & 0.0  & 0.0  & 0.0  & 0.0  & -574.68  \\
1.0  & 0.42  & 1.34  & 1.48  & 1.65  & 0.94  & 0.07  & 1.64  & -0.01  & -0.0  & -0.0  & -0.0  & -0.0  & -0.0  & 0.01  & 0.0  & 0.0  & 0.0  & 0.0  & 0.0  & 0.12  \\
0.0  & 146.0  & -111.07  & -121.73  & -118.8  & 23.73  & \colorbox{yellow}{254.2 }  & -142.73  & 1.07  & -1.0  & 0.0  & 0.0  & 0.0  & 0.0  & -1.07  & 1.0  & 0.0  & 0.0  & 0.0  & 0.0  & 13.6  \\
0.0  & 27.83  & -127.59  & -81.98  & -163.9  & -80.19  & 220.93  & -316.64  & 1.26  & 0.0  & -1.0  & 0.0  & 0.0  & 0.0  & -1.26  & 0.0  & 1.0  & 0.0  & 0.0  & 0.0  & 23.63  \\
0.0  & 89.58  & -271.74  & -331.88  & -319.45  & -77.54  & 185.13  & -331.04  & 1.41  & 0.0  & 0.0  & -1.0  & 0.0  & 0.0  & -1.41  & 0.0  & 0.0  & 1.0  & 0.0  & 0.0  & 213.48  \\
0.0  & 39.58  & -163.34  & 10.52  & -18.65  & -5.94  & 145.93  & -12.64  & 1.01  & 0.0  & 0.0  & 0.0  & -1.0  & 0.0  & -1.01  & 0.0  & 0.0  & 0.0  & 1.0  & 0.0  & 134.88  \\
0.0  & 81.58  & -66.14  & -147.28  & -182.25  & 95.86  & 251.33  & 40.56  & 0.81  & 0.0  & 0.0  & 0.0  & 0.0  & -1.0  & -0.81  & 0.0  & 0.0  & 0.0  & 0.0  & 1.0  & 189.08 
\end{pmatrix}
}
\]

разрешающий столбец = 7

разрешающая строка = 3

разрешающий элемент = 254.2

\[
\scalebox{0.7}{
\setlength{\arraycolsep}{2pt}
\renewcommand{\arraystretch}{1}
\begin{pmatrix}
0.0  & 222.81  & 277.81  & 165.9  & 308.81  & 142.81  & 0.0  & 168.71  & -1.1  & -3.16  & 1.0  & 1.0  & 1.0  & 1.0  & 2.1  & 4.16  & 0.0  & 0.0  & 0.0  & 0.0  & -518.1  \\
1.0  & 0.38  & 1.37  & 1.51  & 1.68  & 0.93  & 0.0  & 1.68  & -0.01  & 0.0  & -0.0  & -0.0  & -0.0  & -0.0  & 0.01  & -0.0  & 0.0  & 0.0  & 0.0  & 0.0  & 0.11  \\
0.0  & 0.57  & -0.44  & -0.48  & -0.47  & 0.09  & 1.0  & -0.56  & 0.0  & -0.0  & 0.0  & 0.0  & 0.0  & 0.0  & -0.0  & 0.0  & 0.0  & 0.0  & 0.0  & 0.0  & 0.05  \\
0.0  & -99.06  & -31.06  & 23.82  & -60.65  & -100.82  & 0.0  & -192.59  & 0.33  & \colorbox{yellow}{0.87 }  & -1.0  & 0.0  & 0.0  & 0.0  & -0.33  & -0.87  & 1.0  & 0.0  & 0.0  & 0.0  & 11.81  \\
0.0  & -16.75  & -190.85  & -243.22  & -232.93  & -94.82  & 0.0  & -227.09  & 0.63  & 0.73  & 0.0  & -1.0  & 0.0  & 0.0  & -0.63  & -0.73  & 0.0  & 1.0  & 0.0  & 0.0  & 203.58  \\
0.0  & -44.23  & -99.58  & 80.41  & 49.55  & -19.56  & 0.0  & 69.3  & 0.39  & 0.57  & 0.0  & 0.0  & -1.0  & 0.0  & -0.39  & -0.57  & 0.0  & 0.0  & 1.0  & 0.0  & 127.08  \\
0.0  & -62.77  & 43.68  & -26.92  & -64.79  & 72.4  & 0.0  & 181.68  & -0.25  & 0.99  & 0.0  & 0.0  & 0.0  & -1.0  & 0.25  & -0.99  & 0.0  & 0.0  & 0.0  & 1.0  & 175.64 
\end{pmatrix}
}
\]

разрешающий столбец = 10

разрешающая строка = 4

разрешающий элемент = 0.8691319171256229

\[
\scalebox{0.7}{
\setlength{\arraycolsep}{2pt}
\renewcommand{\arraystretch}{1}
\begin{pmatrix}
0.0  & -137.38  & 164.88  & 252.53  & 88.29  & -223.77  & 0.0  & -531.57  & 0.09  & 0.0  & -2.64  & 1.0  & 1.0  & 1.0  & 0.91  & 1.0  & 3.64  & 0.0  & 0.0  & 0.0  & -475.15  \\
1.0  & 0.41  & 1.38  & 1.5  & 1.7  & 0.96  & 0.0  & \colorbox{yellow}{1.74 }  & -0.01  & 0.0  & 0.0  & -0.0  & -0.0  & -0.0  & 0.01  & 0.0  & -0.0  & 0.0  & 0.0  & 0.0  & 0.11  \\
0.0  & 0.13  & -0.58  & -0.37  & -0.74  & -0.36  & 1.0  & -1.43  & 0.01  & 0.0  & -0.0  & 0.0  & 0.0  & 0.0  & -0.01  & 0.0  & 0.0  & 0.0  & 0.0  & 0.0  & 0.11  \\
0.0  & -113.98  & -35.73  & 27.41  & -69.78  & -116.0  & 0.0  & -221.59  & 0.38  & 1.0  & -1.15  & 0.0  & 0.0  & 0.0  & -0.38  & -1.0  & 1.15  & 0.0  & 0.0  & 0.0  & 13.59  \\
0.0  & 66.26  & -164.82  & -263.18  & -182.11  & -10.34  & 0.0  & -65.71  & 0.35  & 0.0  & 0.84  & -1.0  & 0.0  & 0.0  & -0.35  & 0.0  & -0.84  & 1.0  & 0.0  & 0.0  & 193.68  \\
0.0  & 21.2  & -79.06  & 64.67  & 89.61  & 47.03  & 0.0  & 196.51  & 0.18  & 0.0  & 0.66  & 0.0  & -1.0  & 0.0  & -0.18  & 0.0  & -0.66  & 0.0  & 1.0  & 0.0  & 119.27  \\
0.0  & 49.92  & 79.01  & -54.02  & 4.2  & 187.08  & 0.0  & 400.77  & -0.62  & 0.0  & 1.14  & 0.0  & 0.0  & -1.0  & 0.62  & 0.0  & -1.14  & 0.0  & 0.0  & 1.0  & 162.2 
\end{pmatrix}
}
\]

разрешающий столбец = 8

разрешающая строка = 2

разрешающий элемент = 1.73999195332931

\[
\scalebox{0.7}{
\setlength{\arraycolsep}{2pt}
\renewcommand{\arraystretch}{1}
\begin{pmatrix}
305.5  & -12.65  & 585.67  & 711.55  & 607.48  & 70.46  & 0.0  & 0.0  & -1.72  & 0.0  & -2.54  & 1.0  & 1.0  & 1.0  & 2.72  & 1.0  & 3.54  & 0.0  & 0.0  & 0.0  & -441.69  \\
0.57  & \colorbox{yellow}{0.23 }  & 0.79  & 0.86  & 0.98  & 0.55  & 0.0  & 1.0  & -0.0  & 0.0  & 0.0  & -0.0  & -0.0  & -0.0  & 0.0  & 0.0  & -0.0  & 0.0  & 0.0  & 0.0  & 0.06  \\
0.82  & 0.46  & 0.56  & 0.87  & 0.66  & 0.43  & 1.0  & 0.0  & 0.0  & 0.0  & -0.0  & 0.0  & 0.0  & 0.0  & -0.0  & 0.0  & 0.0  & 0.0  & 0.0  & 0.0  & 0.2  \\
127.35  & -61.98  & 139.68  & 218.76  & 146.65  & 6.65  & 0.0  & 0.0  & -0.38  & 1.0  & -1.11  & 0.0  & 0.0  & 0.0  & 0.38  & -1.0  & 1.11  & 0.0  & 0.0  & 0.0  & 27.54  \\
37.76  & 81.68  & -112.81  & -206.44  & -117.93  & 26.03  & 0.0  & 0.0  & 0.13  & 0.0  & 0.85  & -1.0  & 0.0  & 0.0  & -0.13  & 0.0  & -0.85  & 1.0  & 0.0  & 0.0  & 197.82  \\
-112.94  & -24.91  & -234.62  & -105.02  & -102.32  & -61.74  & 0.0  & 0.0  & 0.85  & 0.0  & 0.63  & 0.0  & -1.0  & 0.0  & -0.85  & 0.0  & -0.63  & 0.0  & 1.0  & 0.0  & 106.9  \\
-230.33  & -44.12  & -238.25  & -400.09  & -387.23  & -34.74  & 0.0  & 0.0  & 0.74  & 0.0  & 1.07  & 0.0  & 0.0  & -1.0  & -0.74  & 0.0  & -1.07  & 0.0  & 0.0  & 1.0  & 136.97 
\end{pmatrix}
}
\]

разрешающий столбец = 2

разрешающая строка = 2

разрешающий элемент = 0.23463784033759177

\[
\scalebox{0.7}{
\setlength{\arraycolsep}{2pt}
\renewcommand{\arraystretch}{1}
\begin{pmatrix}
336.49  & 0.0  & 628.36  & 758.12  & 660.15  & 100.3  & 0.0  & 53.92  & -1.9  & 0.0  & -2.53  & 1.0  & 1.0  & 1.0  & 2.9  & 1.0  & 3.53  & 0.0  & 0.0  & 0.0  & -438.29  \\
2.45  & 1.0  & 3.37  & 3.68  & 4.16  & 2.36  & 0.0  & 4.26  & -0.01  & 0.0  & 0.0  & -0.0  & -0.0  & -0.0  & 0.01  & 0.0  & -0.0  & 0.0  & 0.0  & 0.0  & 0.27  \\
-0.31  & 0.0  & -1.0  & -0.83  & -1.27  & -0.66  & 1.0  & -1.97  & 0.01  & 0.0  & -0.0  & 0.0  & 0.0  & 0.0  & -0.01  & 0.0  & 0.0  & 0.0  & 0.0  & 0.0  & 0.07  \\
279.17  & 0.0  & 348.79  & 446.87  & 404.66  & 152.87  & 0.0  & 264.16  & -1.28  & 1.0  & -1.07  & 0.0  & 0.0  & 0.0  & 1.28  & -1.0  & 1.07  & 0.0  & 0.0  & 0.0  & 44.17  \\
-162.3  & 0.0  & -388.37  & -507.04  & -457.92  & -166.65  & 0.0  & -348.1  & 1.32  & 0.0  & 0.79  & -1.0  & 0.0  & 0.0  & -1.32  & 0.0  & -0.79  & 1.0  & 0.0  & 0.0  & 175.9  \\
-51.92  & 0.0  & -150.58  & -13.34  & 1.37  & -2.98  & 0.0  & 106.16  & 0.48  & 0.0  & 0.64  & 0.0  & -1.0  & 0.0  & -0.48  & 0.0  & -0.64  & 0.0  & 1.0  & 0.0  & 113.59  \\
-122.27  & 0.0  & -89.41  & -237.74  & -203.6  & 69.32  & 0.0  & 188.02  & 0.1  & 0.0  & \colorbox{yellow}{1.1 }  & 0.0  & 0.0  & -1.0  & -0.1  & 0.0  & -1.1  & 0.0  & 0.0  & 1.0  & 148.8 
\end{pmatrix}
}
\]

разрешающий столбец = 11

разрешающая строка = 7

разрешающий элемент = 1.1007021433850703

\[
\scalebox{0.7}{
\setlength{\arraycolsep}{2pt}
\renewcommand{\arraystretch}{1}
\begin{pmatrix}
54.94  & 0.0  & 422.48  & 210.69  & 191.34  & 259.93  & 0.0  & 486.86  & -1.67  & 0.0  & 0.0  & 1.0  & 1.0  & -1.3  & 2.67  & 1.0  & 1.0  & 0.0  & 0.0  & 2.3  & -95.64  \\
2.53  & 1.0  & 3.43  & 3.84  & 4.3  & 2.31  & 0.0  & 4.14  & -0.01  & 0.0  & 0.0  & -0.0  & -0.0  & 0.0  & 0.01  & 0.0  & 0.0  & 0.0  & 0.0  & -0.0  & 0.17  \\
-0.82  & 0.0  & -1.38  & -1.83  & -2.12  & -0.37  & 1.0  & -1.18  & 0.01  & 0.0  & 0.0  & 0.0  & 0.0  & -0.0  & -0.01  & 0.0  & 0.0  & 0.0  & 0.0  & 0.0  & 0.7  \\
160.71  & 0.0  & 262.17  & 216.55  & 207.42  & 220.03  & 0.0  & 446.31  & -1.18  & 1.0  & 0.0  & 0.0  & 0.0  & -0.97  & 1.18  & -1.0  & 0.0  & 0.0  & 0.0  & 0.97  & 188.33  \\
-74.65  & 0.0  & -324.28  & -336.62  & -311.98  & -216.34  & 0.0  & -482.87  & \colorbox{yellow}{1.24 }  & 0.0  & 0.0  & -1.0  & 0.0  & 0.72  & -1.24  & 0.0  & 0.0  & 1.0  & 0.0  & -0.72  & 69.24  \\
19.71  & 0.0  & -98.2  & 125.94  & 120.65  & -43.59  & 0.0  & -3.99  & 0.42  & 0.0  & 0.0  & 0.0  & -1.0  & 0.59  & -0.42  & 0.0  & 0.0  & 0.0  & 1.0  & -0.59  & 26.41  \\
-111.09  & 0.0  & -81.23  & -215.99  & -184.97  & 62.98  & 0.0  & 170.81  & 0.09  & 0.0  & 1.0  & 0.0  & 0.0  & -0.91  & -0.09  & 0.0  & -1.0  & 0.0  & 0.0  & 0.91  & 135.19 
\end{pmatrix}
}
\]

разрешающий столбец = 9

разрешающая строка = 5

разрешающий элемент = 1.2428515471993733

\[
\scalebox{0.7}{
\setlength{\arraycolsep}{2pt}
\renewcommand{\arraystretch}{1}
\begin{pmatrix}
-45.18  & 0.0  & -12.45  & -240.8  & -227.11  & -30.23  & 0.0  & -160.78  & 0.0  & 0.0  & 0.0  & -0.34  & 1.0  & -0.34  & 1.0  & 1.0  & 1.0  & 1.34  & 0.0  & 1.34  & -2.78  \\
1.65  & 1.0  & -0.38  & -0.12  & 0.63  & -0.23  & 0.0  & -1.54  & 0.0  & 0.0  & 0.0  & -0.01  & 0.0  & 0.01  & 0.0  & 0.0  & 0.0  & 0.01  & 0.0  & -0.01  & 0.98  \\
-0.34  & 0.0  & 0.7  & 0.32  & -0.13  & 1.01  & 1.0  & 1.91  & 0.0  & 0.0  & 0.0  & 0.01  & 0.0  & -0.01  & 0.0  & 0.0  & 0.0  & -0.01  & 0.0  & 0.01  & 0.26  \\
89.89  & 0.0  & -45.48  & -102.81  & -88.57  & 14.78  & 0.0  & -11.8  & 0.0  & 1.0  & 0.0  & -0.95  & 0.0  & -0.29  & 0.0  & -1.0  & 0.0  & 0.95  & 0.0  & 0.29  & 254.02  \\
-60.06  & 0.0  & -260.91  & -270.85  & -251.02  & -174.07  & 0.0  & -388.52  & 1.0  & 0.0  & 0.0  & -0.8  & 0.0  & 0.58  & -1.0  & 0.0  & 0.0  & 0.8  & 0.0  & -0.58  & 55.71  \\
45.18  & 0.0  & 12.45  & \colorbox{yellow}{240.8 }  & 227.11  & 30.23  & 0.0  & 160.78  & 0.0  & 0.0  & 0.0  & 0.34  & -1.0  & 0.34  & 0.0  & 0.0  & 0.0  & -0.34  & 1.0  & -0.34  & 2.78  \\
-105.47  & 0.0  & -56.85  & -190.68  & -161.51  & 79.25  & 0.0  & 207.12  & 0.0  & 0.0  & 1.0  & 0.08  & 0.0  & -0.96  & 0.0  & 0.0  & -1.0  & -0.08  & 0.0  & 0.96  & 129.98 
\end{pmatrix}
}
\]

разрешающий столбец = 4

разрешающая строка = 6

разрешающий элемент = 240.8032056188375

\[
\scalebox{0.7}{
\setlength{\arraycolsep}{2pt}
\renewcommand{\arraystretch}{1}
\begin{pmatrix}
-0.0  & 0.0  & -0.0  & 0.0  & 0.0  & 0.0  & 0.0  & -0.0  & 0.0  & 0.0  & 0.0  & 0.0  & 0.0  & 0.0  & 1.0  & 1.0  & 1.0  & 1.0  & 1.0  & 1.0  & -0.0  \\
1.68  & 1.0  & -0.37  & 0.0  & 0.74  & -0.22  & 0.0  & -1.46  & 0.0  & 0.0  & 0.0  & -0.01  & -0.0  & 0.01  & 0.0  & 0.0  & 0.0  & 0.01  & 0.0  & -0.01  & 0.98  \\
-0.4  & 0.0  & 0.68  & 0.0  & -0.43  & 0.97  & 1.0  & 1.69  & 0.0  & 0.0  & 0.0  & 0.01  & 0.0  & -0.01  & 0.0  & 0.0  & 0.0  & -0.01  & -0.0  & 0.01  & 0.25  \\
109.18  & 0.0  & -40.16  & 0.0  & 8.39  & 27.69  & 0.0  & 56.84  & 0.0  & 1.0  & 0.0  & -0.8  & -0.43  & -0.14  & 0.0  & -1.0  & 0.0  & 0.8  & 0.43  & 0.14  & 255.2  \\
-9.24  & 0.0  & -246.91  & 0.0  & 4.42  & -140.07  & 0.0  & -207.68  & 1.0  & 0.0  & 0.0  & -0.42  & -1.12  & 0.96  & -1.0  & 0.0  & 0.0  & 0.42  & 1.12  & -0.96  & 58.84  \\
0.19  & 0.0  & 0.05  & 1.0  & 0.94  & 0.13  & 0.0  & 0.67  & 0.0  & 0.0  & 0.0  & 0.0  & -0.0  & 0.0  & 0.0  & 0.0  & 0.0  & -0.0  & 0.0  & -0.0  & 0.01  \\
-69.69  & 0.0  & -46.99  & 0.0  & 18.32  & 103.18  & 0.0  & 334.43  & 0.0  & 0.0  & 1.0  & 0.35  & -0.79  & -0.69  & 0.0  & 0.0  & -1.0  & -0.35  & 0.79  & 0.69  & 132.19 
\end{pmatrix}
}
\]

В первой строке не осталось отрицательных элементов (не считая значение целевой функции) И \(u = 0\), значит найдено оптимальное решение для вспомогательной задачи, но начальное угловое и допустимое решение для исходной двойственной задачи.

\underline{Оптимальное решение:}

\[
\scalebox{1}{
\setlength{\arraycolsep}{2pt}
\renewcommand{\arraystretch}{1}
\begin{pmatrix}
0.0  & 0.98  & 0.0  & 0.01  & 0.0  & 0.0  & 0.25  & 0.0  & 58.84  & 255.2  & 132.19  & 0.0  & 0.0  & 0.0 
\end{pmatrix}
}
\]

\section*{Решение двойственной задачи}
Составим симплекс-таблицу для двойственной задачи. Из прошлой матрицы убираем столбцы, соответствующие вектору \(u\), первуюстроку заменяем на расширенный вектор \(b\) и значение целевой функции приравниваем к нулю.
\[
\scalebox{0.6}{
\setlength{\arraycolsep}{2pt}
\renewcommand{\arraystretch}{1}
\begin{pmatrix}
158.0  & 170.0  & 215.0  & 256.0  & 143.0  & 180.0  & 295.0  & 87.0  & 0.0  & 0.0  & 0.0  & 0.0  & 0.0  & 0.0  & 0.0  \\
1.68  & 1.0  & -0.37  & 0.0  & 0.74  & -0.22  & 0.0  & -1.46  & 0.0  & 0.0  & 0.0  & -0.01  & -0.0  & 0.01  & 0.98  \\
-0.4  & 0.0  & 0.68  & 0.0  & -0.43  & 0.97  & 1.0  & 1.69  & 0.0  & 0.0  & 0.0  & 0.01  & 0.0  & -0.01  & 0.25  \\
109.18  & 0.0  & -40.16  & 0.0  & 8.39  & 27.69  & 0.0  & 56.84  & 0.0  & 1.0  & 0.0  & -0.8  & -0.43  & -0.14  & 255.2  \\
-9.24  & 0.0  & -246.91  & 0.0  & 4.42  & -140.07  & 0.0  & -207.68  & 1.0  & 0.0  & 0.0  & -0.42  & -1.12  & 0.96  & 58.84  \\
0.19  & 0.0  & 0.05  & 1.0  & 0.94  & 0.13  & 0.0  & 0.67  & 0.0  & 0.0  & 0.0  & 0.0  & -0.0  & 0.0  & 0.01  \\
-69.69  & 0.0  & -46.99  & 0.0  & 18.32  & 103.18  & 0.0  & 334.43  & 0.0  & 0.0  & 1.0  & 0.35  & -0.79  & -0.69  & 132.19 
\end{pmatrix}
}
\]

Выделяем базисные столбцы с помощью элементарных преобразований строк матрицы.
\[
\scalebox{0.6}{
\setlength{\arraycolsep}{2pt}
\renewcommand{\arraystretch}{1}
\begin{pmatrix}
-55.64  & 0.0  & 64.52  & 0.0  & -98.5  & -102.81  & 0.0  & -334.83  & 0.0  & 0.0  & 0.0  & -0.15  & 0.75  & 0.79  & -244.26  \\
1.68  & 1.0  & -0.37  & 0.0  & 0.74  & -0.22  & 0.0  & -1.46  & 0.0  & 0.0  & 0.0  & -0.01  & -0.0  & 0.01  & 0.98  \\
-0.4  & 0.0  & 0.68  & 0.0  & -0.43  & 0.97  & 1.0  & 1.69  & 0.0  & 0.0  & 0.0  & 0.01  & 0.0  & -0.01  & 0.25  \\
109.18  & 0.0  & -40.16  & 0.0  & 8.39  & 27.69  & 0.0  & 56.84  & 0.0  & 1.0  & 0.0  & -0.8  & -0.43  & -0.14  & 255.2  \\
-9.24  & 0.0  & -246.91  & 0.0  & 4.42  & -140.07  & 0.0  & -207.68  & 1.0  & 0.0  & 0.0  & -0.42  & -1.12  & 0.96  & 58.84  \\
0.19  & 0.0  & 0.05  & 1.0  & 0.94  & 0.13  & 0.0  & 0.67  & 0.0  & 0.0  & 0.0  & 0.0  & -0.0  & 0.0  & 0.01  \\
-69.69  & 0.0  & -46.99  & 0.0  & 18.32  & 103.18  & 0.0  & 334.43  & 0.0  & 0.0  & 1.0  & 0.35  & -0.79  & -0.69  & 132.19 
\end{pmatrix}
}
\]

Начальное угловое решение

\[
\scalebox{1}{
\setlength{\arraycolsep}{2pt}
\renewcommand{\arraystretch}{1}
\begin{pmatrix}
0.0  & 0.98  & 0.0  & 0.01  & 0.0  & 0.0  & 0.25  & 0.0  & 58.84  & 255.2  & 132.19  & 0.0  & 0.0  & 0.0 
\end{pmatrix}
}
\]

\[
\scalebox{0.6}{
\setlength{\arraycolsep}{2pt}
\renewcommand{\arraystretch}{1}
\begin{pmatrix}
38.46  & 0.0  & 90.46  & 501.47  & 374.45  & -39.86  & 0.0  & 0.0  & 0.0  & 0.0  & 0.0  & 0.57  & -1.33  & 1.5  & -238.47  \\
2.09  & 1.0  & -0.26  & 2.19  & 2.81  & 0.06  & 0.0  & 0.0  & 0.0  & 0.0  & 0.0  & -0.01  & -0.01  & 0.01  & 1.01  \\
-0.88  & 0.0  & 0.55  & -2.53  & -2.82  & 0.66  & 1.0  & 0.0  & 0.0  & 0.0  & 0.0  & 0.0  & 0.01  & -0.01  & 0.22  \\
93.21  & 0.0  & -44.56  & -85.13  & -71.9  & 17.0  & 0.0  & 0.0  & 0.0  & 1.0  & 0.0  & -0.92  & -0.07  & -0.26  & 254.22  \\
49.12  & 0.0  & -230.82  & 311.04  & 297.77  & -101.02  & 0.0  & 0.0  & 1.0  & 0.0  & 0.0  & 0.02  & -2.42  & 1.4  & 62.43  \\
0.28  & 0.0  & 0.08  & 1.5  & 1.41  & \colorbox{yellow}{0.19 }  & 0.0  & 1.0  & 0.0  & 0.0  & 0.0  & 0.0  & -0.01  & 0.0  & 0.02  \\
-163.68  & 0.0  & -72.89  & -500.88  & -454.07  & 40.31  & 0.0  & 0.0  & 0.0  & 0.0  & 1.0  & -0.36  & 1.29  & -1.4  & 126.4 
\end{pmatrix}
}
\]

разрешающий столбец = 6

разрешающая строка = 6

разрешающий элемент = 0.18801041695836004

\[
\scalebox{0.6}{
\setlength{\arraycolsep}{2pt}
\renewcommand{\arraystretch}{1}
\begin{pmatrix}
98.03  & 0.0  & 106.88  & 818.97  & 673.89  & 0.0  & 0.0  & 211.99  & 0.0  & 0.0  & 0.0  & 1.02  & -2.65  & 1.95  & -234.81  \\
2.0  & 1.0  & -0.28  & 1.71  & 2.36  & 0.0  & 0.0  & -0.32  & 0.0  & 0.0  & 0.0  & -0.01  & -0.01  & 0.01  & 1.0  \\
-1.86  & 0.0  & 0.28  & -7.76  & -7.74  & 0.0  & 1.0  & -3.49  & 0.0  & 0.0  & 0.0  & -0.01  & \colorbox{yellow}{0.03 }  & -0.02  & 0.16  \\
67.8  & 0.0  & -51.57  & -220.55  & -199.61  & 0.0  & 0.0  & -90.42  & 0.0  & 1.0  & 0.0  & -1.12  & 0.49  & -0.46  & 252.66  \\
200.13  & 0.0  & -189.19  & 1115.79  & 1056.75  & 0.0  & 0.0  & 537.32  & 1.0  & 0.0  & 0.0  & 1.16  & -5.76  & 2.54  & 71.72  \\
1.49  & 0.0  & 0.41  & 7.97  & 7.51  & 1.0  & 0.0  & 5.32  & 0.0  & 0.0  & 0.0  & 0.01  & -0.03  & 0.01  & 0.09  \\
-223.93  & 0.0  & -89.5  & -821.98  & -756.9  & 0.0  & 0.0  & -214.39  & 0.0  & 0.0  & 1.0  & -0.82  & 2.62  & -1.86  & 122.7 
\end{pmatrix}
}
\]

разрешающий столбец = 13

разрешающая строка = 3

разрешающий элемент = 0.03354165301368476

\[
\scalebox{0.6}{
\setlength{\arraycolsep}{2pt}
\renewcommand{\arraystretch}{1}
\begin{pmatrix}
-48.8  & 0.0  & 128.8  & 206.61  & 62.6  & 0.0  & 78.94  & -63.26  & 0.0  & 0.0  & 0.0  & 0.62  & 0.0  & 0.35  & -222.04  \\
1.58  & 1.0  & -0.22  & -0.04  & 0.6  & 0.0  & 0.23  & -1.11  & 0.0  & 0.0  & 0.0  & -0.01  & 0.0  & 0.01  & 1.04  \\
-55.45  & 0.0  & 8.28  & -231.27  & -230.87  & 0.0  & 29.81  & -103.95  & 0.0  & 0.0  & 0.0  & -0.15  & 1.0  & -0.6  & 4.82  \\
94.91  & 0.0  & -55.62  & -107.47  & -86.73  & 0.0  & -14.58  & -39.59  & 0.0  & 1.0  & 0.0  & -1.04  & 0.0  & -0.16  & 250.3  \\
-119.2  & 0.0  & -141.53  & -215.96  & -272.67  & 0.0  & 171.68  & -61.29  & 1.0  & 0.0  & 0.0  & 0.29  & 0.0  & -0.93  & 99.49  \\
-0.34  & 0.0  & 0.69  & 0.32  & -0.12  & 1.0  & 0.99  & \colorbox{yellow}{1.88 }  & 0.0  & 0.0  & 0.0  & 0.01  & 0.0  & -0.01  & 0.25  \\
-78.55  & 0.0  & -111.2  & -215.67  & -151.65  & 0.0  & -78.16  & 58.14  & 0.0  & 0.0  & 1.0  & -0.42  & 0.0  & -0.28  & 110.05 
\end{pmatrix}
}
\]

разрешающий столбец = 8

разрешающая строка = 6

разрешающий элемент = 1.8798845470692727

\[
\scalebox{0.6}{
\setlength{\arraycolsep}{2pt}
\renewcommand{\arraystretch}{1}
\begin{pmatrix}
-60.23  & 0.0  & 151.88  & 217.22  & 58.42  & 33.65  & 112.13  & 0.0  & 0.0  & 0.0  & 0.0  & 0.83  & 0.0  & 0.06  & -213.57  \\
\colorbox{yellow}{1.38 }  & 1.0  & 0.18  & 0.14  & 0.53  & 0.59  & 0.81  & 0.0  & 0.0  & 0.0  & 0.0  & -0.01  & 0.0  & 0.0  & 1.19  \\
-74.24  & 0.0  & 46.21  & -213.83  & -237.74  & 55.3  & 84.35  & 0.0  & 0.0  & 0.0  & 0.0  & 0.2  & 1.0  & -1.08  & 18.73  \\
87.76  & 0.0  & -41.17  & -100.83  & -89.35  & 21.06  & 6.19  & 0.0  & 0.0  & 1.0  & 0.0  & -0.91  & 0.0  & -0.34  & 255.6  \\
-130.28  & 0.0  & -119.16  & -205.68  & -276.73  & 32.6  & 203.84  & 0.0  & 1.0  & 0.0  & 0.0  & 0.5  & 0.0  & -1.21  & 107.69  \\
-0.18  & 0.0  & 0.36  & 0.17  & -0.07  & 0.53  & 0.52  & 1.0  & 0.0  & 0.0  & 0.0  & 0.0  & 0.0  & -0.0  & 0.13  \\
-68.04  & 0.0  & -132.42  & -225.42  & -147.81  & -30.93  & -108.67  & 0.0  & 0.0  & 0.0  & 1.0  & -0.62  & 0.0  & -0.01  & 102.27 
\end{pmatrix}
}
\]

разрешающий столбец = 1

разрешающая строка = 2

разрешающий элемент = 1.376213534900201

\[
\scalebox{0.6}{
\setlength{\arraycolsep}{2pt}
\renewcommand{\arraystretch}{1}
\begin{pmatrix}
0.0  & 43.77  & 159.96  & 223.45  & 81.68  & 59.42  & 147.46  & 0.0  & 0.0  & 0.0  & 0.0  & 0.54  & 0.0  & 0.15  & -161.58  \\
1.0  & 0.73  & 0.13  & 0.1  & 0.39  & 0.43  & 0.59  & 0.0  & 0.0  & 0.0  & 0.0  & -0.0  & 0.0  & 0.0  & 0.86  \\
0.0  & 53.95  & 56.16  & -206.16  & -209.08  & 87.05  & 127.9  & 0.0  & 0.0  & 0.0  & 0.0  & -0.16  & 1.0  & -0.97  & 82.81  \\
0.0  & -63.77  & -52.94  & -109.9  & -123.24  & -16.48  & -45.28  & 0.0  & 0.0  & 1.0  & 0.0  & -0.49  & 0.0  & -0.47  & 179.85  \\
0.0  & 94.66  & -101.69  & -192.21  & -226.42  & 88.33  & 280.25  & 0.0  & 1.0  & 0.0  & 0.0  & -0.13  & 0.0  & -1.02  & 220.14  \\
0.0  & 0.13  & 0.39  & 0.19  & 0.0  & 0.61  & 0.63  & 1.0  & 0.0  & 0.0  & 0.0  & 0.0  & 0.0  & -0.0  & 0.29  \\
0.0  & 49.44  & -123.29  & -218.38  & -121.53  & -1.83  & -68.76  & 0.0  & 0.0  & 0.0  & 1.0  & -0.95  & 0.0  & 0.09  & 161.0 
\end{pmatrix}
}
\]

В первой строке не осталось отрицательных элементов, значит найдено оптимальное решение.

\underline{Оптимальное решение:}

\[
\scalebox{1}{
\setlength{\arraycolsep}{2pt}
\renewcommand{\arraystretch}{1}
\begin{pmatrix}
0.86  & 0.0  & 0.0  & 0.0  & 0.0  & 0.0  & 0.0  & 0.29 
\end{pmatrix}
}
\]

\underline{Целевая функция} = 161.5833133088462

\section*{Ответ:}
Оптимальное решение:
\[
\scalebox{1}{
\setlength{\arraycolsep}{2pt}
\renewcommand{\arraystretch}{1}
\begin{pmatrix}
0.0  & 0.0  & 0.0  & 0.54  & 0.0  & 0.15 
\end{pmatrix}
}
\]

Целевая функция прямой задачи = 161.58331330884624


Оптимальное решение:
\[
\scalebox{1}{
\setlength{\arraycolsep}{2pt}
\renewcommand{\arraystretch}{1}
\begin{pmatrix}
0.86  & 0.0  & 0.0  & 0.0  & 0.0  & 0.0  & 0.0  & 0.29 
\end{pmatrix}
}
\]

Целевая функция двойственной задачи = -161.5833133088462
\section*{Листинг}
\lstinputlisting[language=Python]{main.py}
\end{document}
